\chapter{Nombres Complexes}
\section{Le plan complexe}

L'ensemble des nombres complexes, noté $\mathbb{C}$ est l'ensemble des couples de points $(x, y) \in \mathbb{R}^2$ suivant les lois:
\begin{align}
\forall\left(z_1, z_2\right) \in \mathbb{C}^2 \mid z_1 & =\left(x_1, y_1\right) \in \mathbb{R}^2, z_2=\left(x_2, y_2\right) \in \mathbb{R}^2 \\
z_1 & +z_2=\left(x_1+x_2, y_1+y_2\right) \\
z_1 & \times z_2=\left(x_1 x_2-y_1 y_2, x_1 y_2+x_2 y_1\right)
\end{align}

Définissonsle nombre $i$ (parfois noté $j$ en physique) comme le nombre tel que $i^2=-1$. Ainsi, le nombre complexe $z=(x, y) \in \mathbb{R}^2$ peut s'écrire $z=x+i y$. $x$ est appelé partie réelle de $z$, notée $\mathcal{R}(z)$ et $y$ est la partie imaginaire de $z$, notée $\mathcal{I}(z)$. C'est la notation cartésienne. Il peut aussi se noter $z = r \exp(i \theta)$, avec $r$ un nombre réel > 0 appelé module de $z$ et noté $|z|$ et $\theta$ un angle défini à $2\pi$ près, appelé argument de $z$. C'est la notation polaire. On passe facilement d'une notation à l'autre via:

\begin{align}
    x &= r \cos \theta\\
    y &= r \sin \theta\\
    r &= \sqrt{x^2+y^2}\\
    \theta &= \arctan2 \left(\frac{y}{x}\right)
\end{align}

Pour un complexe $z=x+i y$, on défini le complexe conjugué de $z$, noté $\bar{z}$ comme $\bar{z}=x-i y$. On a les propriétés suivantes:

\begin{align} 
    & \overline{\left(\overline{z_1}\right)}=z_{1} \\ 
    & \overline{z_1+z_2}=\overline{z}_{1}+\overline{z}_{2} \\ 
    & \overline{z_1 z_2}=\overline{z}_{1} \overline{z}_{2} \\ 
    & \overline{\left(\frac{z_1}{z_2}\right)}=\frac{\overline{z}_{1}}{\overline{z}_{2}} \\ & \overline{z_1^n}=\left(\overline{z}_{1}\right)^{n}
\end{align}


\begin{figure}[H]
    \centering
    \begin{tikzpicture}[scale=2]
        % Draw the real axis
        \draw[->] (-2,0) -- (2,0) node[right] {$\Re$};
        
        % Draw the imaginary axis
        \draw[->] (0,-1.5) -- (0,1.5) node[above] {$\Im$};
        
        % Draw the complex number
        \fill (1.5,1.2) circle (1.5pt);
        \draw (0,0) -- (1.5,1.2) node[midway, above] {$r$} node[right] {$z = r(\cos \theta + i \sin \theta)$};

        \draw[dotted] (1.5,1.2) -- (1.5,0);
        \draw[dotted] (1.5,1.2) -- (0,1.2);
        \node at (-0.15,1.2) {$x$};
        \node at (1.5,-0.15) {$y$};
        % Draw the angle theta
        \draw (0,0) -- (1,0) arc (0:39:1) node[midway, right] {$\theta$};
        
        % Draw the conjugate of the complex number
        \fill (1.5,-1.2) circle (1.5pt);
        \draw (0,0) -- (1.5,-1.2) node[right] {$\overline{z} = r(\cos \theta - i \sin \theta)$};

      \end{tikzpicture}
\end{figure}

L'ensemble des complexes $\mathbb{C}$ est un corps, c'est à dire un type d'ensemble fort sympathique qui admet toutes les opérations qu'on a naturellement envie d'écrire. Soient deux nombres complexes $\left(z_1, z_2\right) \in \mathbb{C}^2$ tels que $z_1=x_1+i y_1$ et $z_2=x_2+i y_2$, avec $\left(x_1, x_2, y_1, y_2\right) \in \mathbb{R}^4$. On a:

\begin{align}
& \text { Si } z_1=z_2, \text { alors } x_{1}=x_{2} \text { et } y_{1}=y_{2} \\
& z_1+z_2=x_{1}+x_{2}+i\left(y_{1}+y_{2}\right) \\
& z_1 \times z_2=\left(x_{1}+i y_{1}\right)\left(x_{2}+i y_{2}\right)=x_{1} x_{2}-y_{1} y_{2}+i\left(x_{1} y_{2}+x_{2} y_{1}\right) \\
& \frac{z_1}{z_2}=\frac{x_{1}+i y_{1}}{x_{2}+i y_{2}}=\frac{x_{1} x_{2}+y_{1} y_{2}}{x_{2}^{2}+y_{2}^{2}}+i \frac{x_{2} y_{1}-x_{1} y_{2}}{x_{2}^{2}+y_{2}^{2}} \\
& \forall \lambda \in \mathbb{R}, \lambda z_1=\lambda x_{1}+i \lambda y_{1}
\end{align}

on a de plus:

\begin{align}
& \bar{z}+z=2 \mathcal{R}(z) \\
& z-\bar{z}=2 i \mathcal{I}(z) \\
& z \bar{z}=x^2+y^2
\end{align}

